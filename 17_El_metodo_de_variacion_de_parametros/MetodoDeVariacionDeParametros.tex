%El método de variación de parámetros
\documentclass[12pt]{article}
\usepackage[total={18cm,22cm},top=1cm, left=2cm]{geometry}
\usepackage[spanish]{babel}
\usepackage[utf8]{inputenc}
\def\TresOp#1#2#3{
\begin{itemize}
\item[A)] #1
\item[B)] #2
\item[C)] #3
\end{itemize}
}
\def\SOH#1{%Second order Homogeneous
#1^{\prime\prime}+p(t)#1^{\prime}+q(t)#1
}
\def\DotPro#1#2#3#4{
#1#2+#3#4
}
\def\LasUs{
u_{1}(t)\mbox{\ y\ }u_{2}(t) 
}
\def\LasYs{
y_{1}(t)\mbox{\ y\ }y_{2}(t)
}
\def\L#1{
#1^{\prime\prime}+#1
}
%REF: Martin Braun, Differential Equations and their Applications, Fourth Edition, page 153,
% Editorial Springer, 1993.
\begin{document}
\section{El método de variación de parámetros}
A continuación describimos un método muy general para encontrar una solución particular $\psi(t)$ 
de la ecuación no homogénea
\begin{equation}
L[y]=\frac{d^{2}y}{dt^{2}}+p(t)\frac{dy}{dt}+q(t)y=g(t),
\label{Ecu01}
\end{equation}
una vez que las soluciones de la ecuación homogénea
\begin{equation}
L[y]=\frac{d^{2}y}{dt^{2}}+p(t)\frac{dy}{dt}+q(t)y=0
\label{Ecu02}
\end{equation}
son conocidas. El principio básico de este método es usar nuestro conocimiento de las soluciones 
de la ecuación homogénea para ayudarnos a encontrar una solución de la ecuación no homogénea.

Sean $y_{1}(t)$ y $y_{2}(t)$ dos soluciones linealmente independientes de la ecuación homogénea 
(\ref{Ecu02}). Trataremos de encontrar una solución particular $\psi(t)$ de la ecuación no 
homogénea (\ref{Ecu01}) de la forma
\begin{equation}
\psi(t)=u_{1}(t)y_{1}(t)+u_{2}(t)y_{2}(t);
\label{Ecu03}
\end{equation}
esto es, trataremos de encontrar las funciones $u_{1}(t)$ y $u_{2}(t)$ tal que la combinación 
lineal $u_{1}(t)y_{1}(t)+u_{2}(t)y_{2}(t)$ es una solución de (\ref{Ecu01}). En este método, se 
buscará encontrar $u_{1}(t)$ y $u_{2}(t)$ como las soluciones de dos ecuaciones de primer orden 
simples. Observemos que la ecuación diferencial (\ref{Ecu01}) impone solamente una condición 
sobre las dos funciones desconocidas $u_{1}(t)$ y $u_{2}(t)$. Por lo tanto, tenemos uan cierta 
``libertad'' en la elección de $u_{1}(t)$ y $u_{2}(t)$. Nuestro objetivo es imponer una condición 
adicional sobre $u_{1}(t)$ y $u_{2}(t)$ la cual hará la expresión $L[u_{1}y_{1}+u_{2}y_{2}]$ tan 
simple como sea posible. Calculando
\begin{eqnarray}
\frac{d}{dt}\psi(t)&=&\frac{d}{dt}[u_{1}y_{1}+u_{2}y_{2}]\nonumber\\
&=&[u_{1}y_{1}^{\prime}+u_{2}y_{2}^{\prime}]+[u_{1}^{\prime}y_{1}+u_{2}^{\prime}y_{2}]\nonumber
\end{eqnarray}
De esta expresión, se puede notar que si ponemos como condición que
\begin{equation}
y_{1}(t)u_{1}^{\prime}(t)+y_{2}(t)u_{2}^{\prime}(t)=0,
\label{Ecu04}
\end{equation}
entonces, $d^{2}\psi/dt^{2}$ y consecuentemente $L[\psi]$, no contendrá derivadas de segundo orden 
de $u_{1}$ y $u_{2}$. Esto sugiere que, imponiendo la condición (\ref{Ecu04}) sobre las funciones 
$u_{1}(t)$ y $u_{2}(t)$, en este caso, entonces,
\begin{eqnarray}
L[\psi]&=&[u_{1}y_{1}^{\prime}+u_{2}y_{2}^{\prime}]^{\prime}+p(t)[u_{1}y_{1}^{\prime}+u_{2}y_{2}^{\prime}]+
q(t)[u_{1}y_{1}+u_{2}y_{2}]\nonumber\\
&=&u_{1}^{\prime}y_{1}^{\prime}+u_{2}^{\prime}y_{2}^{\prime}+
u_{1}[\SOH{y_{1}}]+u_{2}[\SOH{y_{2}}]\nonumber\\
&=&u_{1}^{\prime}y_{1}^{\prime}+u_{2}^{\prime}y_{2}^{\prime}\nonumber
\end{eqnarray}
Dado que $y_{1}(t)$ y $y_{2}(t)$ son soluciones de la ecuación homogénea $L[y]=0$. Consecuentemente, 
$\psi=\DotPro{u_{1}}{y_{1}}{u_{2}}{y_{2}}$ es una solución de la ecuación no homogenea (\ref{Ecu01}) 
si $\LasUs$ satisfacen las dos ecuaciones
\begin{eqnarray}
\DotPro{y_{1}(t)}{u_{1}^{\prime}(t)}{y_{2}(t)}{u_{2}^{\prime}(t)}&=&0\label{Ecu06}\\
\DotPro{y_{1}^{\prime}(t)}{u_{1}^{\prime}(t)}{y_{2}^{\prime}(t)}{u_{2}^{\prime}(t)}&=&g(t)\label{Ecu07}
\end{eqnarray}
Aquí podemos notar que las ecuaciones (\ref{Ecu06}) y (\ref{Ecu07}) pueden rescribirse como
\begin{equation}\left[
\begin{array}{cc}
y_{1}(t) & y_{2}(t)\\
y_{1}^{\prime}(t) & y_{2}^{\prime}(t)
\end{array}\right]\left[\begin{array}{c}
u_{1}^{\prime}(t)\\
u_{2}^{\prime}(t)
\end{array}\right]=\left[\begin{array}{c}
0\\
g(t)
\end{array}\right]
\label{Ecu07}
\end{equation}
y que el determinante de este sistema es el Wronskiano de las soluciones $y_{1}(t)$ y $y_{2}(t)$ de 
$L[y]=0$. De (\ref{Ecu07}), usando el Wronskiano de $\LasYs$, las soluciones para $u_{1}^{\prime}$ y 
$u_{2}^{\prime}$ se pueden calcular como
\begin{equation}
u_{1}^{\prime}(t)=\frac{1}{W(y_{1},y_{2})}\left|\begin{array}{cc}
0    & y_{2}(t)\\
g(t) & y_{2}^{\prime}(t)
\end{array}\right|=-\frac{g(t)y_{2}(t)}{W(y_{1},y_{2})}
\label{Ecu08}
\end{equation}
y
\begin{equation}
u_{1}^{\prime}(t)=\frac{1}{W(y_{1},y_{2})}\left|\begin{array}{cc}
y_{1}(t)         & 0\\
y_{2}^{\prime}(t)& g(t) 
\end{array}\right|=\frac{g(t)y_{1}(t)}{W(y_{1},y_{2})}.
\label{Ecu09}
\end{equation}
Finalmente, obtenemos $u_{1}(t)$ y $u_{2}(t)$ integrando los lados derechos de (\ref{Ecu08}) y 
(\ref{Ecu09}).
\subsection{Comentario}
La solución general de la ecuación homogénea (\ref{Ecu02}) es
\begin{equation}
y(t)=\DotPro{c_{1}}{y_{1}(t)}{c_{2}}{y_{2}(t)}.
\label{Ecu10}
\end{equation}
Permitiendo que $c_{1}$ y $c_{2}$ varíen con el tiempo, obtenemos una solución de la ecuación 
no homogénea. Por esa razón, este método es conocido como el {\bf método de variación de parámetros}. 
\section{Ejemplo}
\begin{itemize}
\item[(a)] Encuentre una solución particular $\psi(t)$ de la ecuación
\begin{equation}
\frac{d^{2}y}{dt^{2}}+y=\tan t
\label{Ecu11}
\end{equation}
sobre el intervalo $-\pi/2<t<\pi/2$.
\item[(b)] Encuentre la solución $y(t)$ de (\ref{Ecu11}) la cual satisface las condiciones iniciales 
$y(0)=1$, $y^{\prime}(0)=1$.
\end{itemize}
{\it Solución.}\\
\begin{itemize}
\item[(a)] Las funciones $y_{1}(t)=\cos t$ y $y_{2}(t)=\sin t$ son dos soluciones linealmente 
independientes de la ecuación homogénea $\L{y}=0$ con
\begin{equation}
W(y_{1},y_{2})=y_{1}y_{2}^{\prime}-y_{1}^{\prime}y_{2}=(\cos t)\sin t-(-\sin t)\cos t=1
\label{Ecu12}
\end{equation}
Entonces, de (\ref{Ecu08}),
\begin{equation}
u_{1}^{\prime}(t)=-\tan t\sin t
\label{Ecu13}
\end{equation}
y de (\ref{Ecu09}),
\begin{equation}
u_{2}^{\prime}(t)=\tan t\cos t
\label{Ecu14}
\end{equation}
Integrando la ecuación (\ref{Ecu13})
\begin{eqnarray}
u_{1}(t)&=&-\int\tan t\sin t\,dt=-\int\frac{\sin^{2}t}{\cos t}dt\nonumber\\
&=&\int\frac{\cos^{2}t-1}{\cos t}dt=\sin t-\ln\left|\sec t+\tan t\right|\nonumber\\
&=&\sin t-\ln(\sec t+\tan t),\,\,\,\,-\frac{\pi}{2}<t<\frac{\pi}{2}
\label{Ecu15}
\end{eqnarray}
mientras que integrando (\ref{Ecu14})
\begin{equation}
u_{2}(t)=\int\tan t\cos t\,dt=\int\sin t\,dt=-\cos t.
\label{Ecu16}
\end{equation}
Consecuentemente,
\begin{eqnarray}
\psi(t)&=&\cos t[\sin t-\ln(\sec t+\tan t)]+\sin t(-\cos t)\nonumber\\
&=&(-\cos t)\ln(\sec t+\tan t)
\label{Ecu17}
\end{eqnarray}
es una solución particular de (\ref{Ecu11}) sobre el intervalo $-\pi/2<t<\pi/2$.
\item[(b)] A la luz de la información anterior, la solución general de (\ref{Ecu01}) está dada por
\begin{equation}
y(t)=c_{1}\cos t+c_{2}\sin t-\cos t\ln(\sec t+\tan t)
\label{Ecu18}
\end{equation}
para alguna elección de las constantes $c_{1}$ y $c_{2}$. Las constantes $c_{1}$ y $c_{2}$ se determinan 
usando las condiciones iniciales
\begin{equation}
c_{1}=y(0)=1,\mbox{\ y\ }c_{2}-1=y^{\prime}(0)=1.
\label{Ecu19}
\end{equation}
Por lo tanto, $c_{1}=1$, $c_{2}=2$ y
\begin{equation}
y(t)=\cos t+2\sin t-\cos t\ln(\sec t+\tan t),\,\,\,\,-\frac{\pi}{2}<t<\frac{\pi}{2}.
\label{Ecu20}
\end{equation}
En resumen, (\ref{Ecu20}) es la solución del problema de valor inicial (\ref{Ecu01}) con condiciones 
iniciales $y(0)=1$, $y^{\prime}(0)=1$; obtenida por el método de variación de parámetros.
\end{itemize}

\newpage
\section{Reactivos}
\begin{enumerate}
\item Sea $L[y]=\SOH{y}$. Entonces $L[y]=\SOH{y}=g(t)$ es
\TresOp{una ecuación diferencial lineal no homogénea}
{una ecuación diferencial lineal homogénea}
{una ecuación diferencial no lineal}
{\tiny R=A}
\item Sea $L[y]=\SOH{y}$, donde $p(t)$ y $q(t)$ son funciones continuas. Si se propone 
$\psi(t)=u_{1}(t)y_{1}(t)+u_{2}(t)y_{2}(t)$, y suponemos que $y_{1}(t)u_{1}^{\prime}(t)+y_{2}(t)u_{2}^{\prime}(t)=0$, ?`a qué es igual $L[\psi]$?
\TresOp{$\SOH{y_{1}}$}
{$\SOH{y_{2}}$}
{$u_{1}^{\prime}y_{1}^{\prime}+u_{2}^{\prime}y_{2}^{\prime}$}
{\tiny R=C}
\item Considerando el sistema de ecuaciones
$$
\left[
\begin{array}{cc}
y_{1}(t) & y_{2}(t)\\
y_{1}^{\prime}(t) & y_{2}^{\prime}(t)
\end{array}\right]\left[\begin{array}{c}
u_{1}^{\prime}(t)\\
u_{2}^{\prime}(t)
\end{array}\right]=\left[\begin{array}{c}
0\\
g(t)
\end{array}\right]
$$
?`A qué es igual el determinante del sistema?
\TresOp{El Wronskiano de $\LasUs$}
{El Wronskiano de $y(t)$ y $g(t)$}
{El Wronskiano de $\LasYs$}
{\tiny R=C}
\item Considerando el sistema de ecuaciones
$$
\left[
\begin{array}{cc}
y_{1}(t) & y_{2}(t)\\
y_{1}^{\prime}(t) & y_{2}^{\prime}(t)
\end{array}\right]\left[\begin{array}{c}
u_{1}^{\prime}(t)\\
u_{2}^{\prime}(t)
\end{array}\right]=\left[\begin{array}{c}
0\\
g(t)
\end{array}\right]
$$
?`Cuál es la solución para $u_{1}^{\prime}(t)$?
\TresOp{$-\frac{g(t)}{W(y_{1},y_{2})}$}
{$-\frac{y_{2}(t)}{W(y_{1},y_{2})}$}
{$-\frac{g(t)y_{2}(t)}{W(y_{1},y_{2})}$}
{\tiny R=C}
\newpage
\item Considerando el sistema de ecuaciones
$$
\left[
\begin{array}{cc}
y_{1}(t) & y_{2}(t)\\
y_{1}^{\prime}(t) & y_{2}^{\prime}(t)
\end{array}\right]\left[\begin{array}{c}
u_{1}^{\prime}(t)\\
u_{2}^{\prime}(t)
\end{array}\right]=\left[\begin{array}{c}
0\\
g(t)
\end{array}\right]
$$
?`Cuál es la solución para $u_{2}^{\prime}(t)$?
\TresOp{$-\frac{g(t)}{W(y_{1},y_{2})}$}
{$\frac{g(t)y_{1}(t)}{W(y_{1},y_{2})}$}
{$-\frac{g(t)y_{2}(t)}{W(y_{1},y_{2})}$}
{\tiny R=B}
\item Si definimos $L[y]=y^{\prime\prime}+y$, la ecuación $y^{\prime\prime}+y=\tan t$ es 
equivalente a
\TresOp{$L[y]=\tan t$}
{$L[y]=0$}
{$L[y]=t$}
{\tiny R=A}
\item ?`Cuál es la ecuación homogénea asociada a la ecuación 
$y^{\prime\prime}+y=\tan t$
\TresOp{$y^{\prime\prime}+y=\tan t$}
{$y^{\prime\prime}+y=t$}
{$y^{\prime\prime}+y=0$}
{\tiny R=C}
\item Definiendo $L[y]=y^{\prime\prime}+y$, ?`cuáles son las raíces de la ecuación característica 
correspondiente a la ecuación $L[y]=0$?
\TresOp{$1+j$, $1-j$}
{$1,-1$}
{$j$, $-j$}
{\tiny R=C}
\newpage
\item La solución de la ecuación $y^{\prime\prime}+y=0$ es de la forma
\TresOp{$c_{1}\cos t+c_{2}\sin t$}
{$e^{t}(c_{1}\cos t+c_{2}\sin t)$}
{$c_{1}e^{t}+c_{2}e^{-t}$}
{\tiny R=A}
\item ?`Cuáles son dos soluciones linealmente independientes de $\L{y}=0$?
\TresOp{$e^{t}$ y $e^{-t}$}
{$\cos t$ y $\sin t$}
{$e^{t}\cos t$ y $e^{t}\sin t$}
{\tiny R=B}
\item Usando el método de variación de parámetros para resolver $\L{y}=\tan t$, ?`a qué es igual 
$u_{1}^{\prime}(t)$?
\TresOp{$\tan t\sin t$}
{$-\tan t\sin t$}
{$-\tan t\cos t$}
{\tiny R=B}
\item Usando el método de variación de parámetros para resolver $\L{y}=\tan t$, ?`a qué es igual 
$u_{2}^{\prime}(t)$?
\TresOp{$\tan t\cos t$}
{$-\tan t\cos t$}
{$-\tan t\sin t$}
{\tiny R=A}
\item Siguiendo el método de variación de parámetros, ?`Cuál es la función $u_{1}(t)$?
\TresOp{$\sin t-\ln(\sec t+\tan t),\,\,\,\,-\frac{\pi}{2}<t<\frac{\pi}{2}$}
{$\sin t -\cos t,\,\,\,\,-\frac{\pi}{2}<t<\frac{\pi}{2}$}
{$\ln(\sec t+\tan t),\,\,\,\,-\frac{\pi}{2}<t<\frac{\pi}{2}$}
{\tiny R=A}
\newpage
\item Siguiendo el método de variación de parámetros, ?`Cuál es la función $u_{2}(t)$?
\TresOp{$\sin t\cos t$}
{$\cos t$}
{$-\cos t$}
{\tiny R=C}
\item Utilizando las dos respuestas anteriores, obtenidas por el método de variación de parámetros; 
?`Cuál es la solución particular de $\L{y}=\tan t$?
\TresOp{$(-\cos t)\ln(\sin t+\tan t),\,\,\,-\frac{\pi}{2}<t<\frac{\pi}{2}$}
{$(-\cos t)\ln(\sec t+\tan t),\,\,\,-\frac{\pi}{2}<t<\frac{\pi}{2}$}
{$(-\sin t)\ln(\sec t+\tan t),\,\,\,-\frac{\pi}{2}<t<\frac{\pi}{2}$}
{\tiny R=B}
\end{enumerate}
\end{document}