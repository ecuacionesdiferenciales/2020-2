\documentclass[12pt]{article}
\usepackage[total={18cm,22cm},top=1cm, left=2cm]{geometry}
\usepackage[spanish]{babel}
\usepackage[utf8]{inputenc}
\def\TresOp#1#2#3{
\begin{itemize}
\item[A)] #1
\item[B)] #2
\item[C)] #3
\end{itemize}
}

\begin{document}
\section{Solución de problemas de valor inicial}
Se verificará que la función $y=3e^{2t}+e^{-2t}-3t$ es una solución del problema de 
valor inicial
\begin{equation}
y^{\prime\prime}-4y=12t,\hspace{2cm}y(0)=4,\,\,y^{\prime}(0)=1.
\label{Ecu0}
\end{equation}


La ecuación diferencial es lineal, los coeficientes y la función $g(t)=12t$ son continuas, 
y $a_{2}(t)=1\ne 0$ sobre cualquier intervalo $I$ que contenga $t=0$. Se concluye, del teorema 
4.1.1 (véase REF Zill, pág. 118) que el problema de valor inicial que estamos considerando 
tiene solución única.
\begin{enumerate}
\item La ecuación $y^{\prime\prime}-4y=12t$ es 
\TresOp{una ecuación diferencial lineal homogénea}
{una ecuación diferencial lineal no homogénea}
{una ecuación diferencial no lineal}
{\tiny R=B}
\item El problema de encontrar $y(t)$ tal que
\begin{equation}
y^{\prime\prime}-4y=12t,\hspace{2cm}y(0)=4,\,\,y^{\prime}(0)=1.
\label{EcuN15}
\end{equation}
es
\TresOp{un problema de valor inicial}
{un problema de solución única}
{un problema de valores en la frontera}
{\tiny R=A}
\item Para resolver el problema de encontrar $y(t)$ tal que
\begin{equation}
y^{\prime\prime}-4y=12t,\hspace{2cm}y(0)=4,\,\,y^{\prime}(0)=1.
\end{equation}
primero se debe resolver
\TresOp{la ecuación diferencial reducible a exacta}
{la ecuación diferencial lineal homogénea asociada}
{la ecuación diferencial dual}
{\tiny R=B}
\newpage
\item La ecuación diferencial $y^{\prime\prime}-4y=0$ tiene asociada la ecuación característica
\TresOp{$s^{2}-4s=0$}
{$s^{2}+4=0$}
{$s^{2}-4=0$}
{\tiny R=C}
\item Las raíces de la ecuación característica asociada con $y^{\prime\prime}-4y=0$ son
\TresOp{$s_{1}=4,\,s_{2}=-4$}
{$s_{1}=2,\,\,s_{2}=2$}
{$s_{1}=2,\,s_{2}=-2$}
{\tiny R=C}
\item La solución de la ecuación $y^{\prime\prime}-4y=0$ es de la forma
\TresOp{$y_{h}(t)=c_{1}e^{2t}+c_{2}te^{2t}$}
{$y_{h}(t)=c_{1}e^{2t}+c_{2}e^{-2t}$}
{$y_{h}(t)=c_{1}e^{2t}+c_{2}te^{-2t}$}
{\tiny R=B}
\item Si definimos $y_{p}(t)=c_{3}t$ (donde $c_{3}$ es una constante), y sustituimos $y=y_{p}(t)$ 
en $y^{\prime\prime}-4y=12t$ se encuentra que la constante $c_{3}$  es igual a
\TresOp{12}
{-4}
{-3}
{\tiny R=C}
\item La solución del problema de encontrar $y(t)$ tal que $y^{\prime\prime}-4y=12t$ sujeto a 
$y(0)=4,\,\,y^{\prime}(0)=1$ es de la forma
\TresOp{$y(t)=c_{1}e^{2t}+c_{2}te^{2t}+c_{3}t$ (donde el valor de $c_{3}$ ya se ha calculado)}
{$y(t)=c_{1}e^{2t}+c_{2}e^{-2t}+c_{3}t$ (donde el valor de $c_{3}$ ya se ha calculado)}
{$y(t)=c_{1}e^{2t}+c_{2}e^{2t}+c_{3}t$ (donde el valor de $c_{3}$ ya se ha calculado)}
{\tiny R=B}
\newpage
\item Sustituyendo $t=0$ y la condición inicial $y(0)=4$ en su respuesta al reactivo anterior 
conduce a la ecuación
\TresOp{$c_{1}+c_{2}=4$}
{$c_{1}+2c_{2}=4$}
{$4c_{1}+c_{2}=4$}
{\tiny R=A}
\item ?`Cuál es el sistema de ecuaciones que se debe resolver para encontrar $c_{1}$ y $c_{2}$ 
de la solución al problema planteado? ($y^{\prime\prime}-4y=12t$ sujeto a $y(0)=4$, 
$y^{\prime}(0)=1$)
\TresOp{$c_{1}+c_{2}=4$, $2c_{1}-2c_{2}=4$}
{$c_{1}+c_{2}=4$, $2c_{1}+2c_{2}=4$}
{$c_{1}+c_{2}=4$, $2c_{1}-2c_{2}=0$}
{\tiny R=A}
\item La solución al problema de encontrar $y(t)$ tal que 
$y^{\prime\prime}-4y=12t$ sujeto a $y(0)=4$, $y^{\prime}(0)=1$ es
\TresOp{$y(t)=3e^{2t}+te^{2t}-3t$}
{$y(t)=3e^{2t}+e^{-2t}-3t$}
{$y(t)=e^{2t}+3e^{-2t}-3t$}
{\tiny R=B}
\end{enumerate}
\subsection{Obtención de la solución a un problema de valor inicial}
Para resolver el problema de valor inicial: encontrar $y(t)$ tal que
\begin{equation}
y^{\prime\prime}-4y=12t,\hspace{2cm}y(0)=4,\,\,y^{\prime}(0)=1.
\end{equation}
primero se encuentra la solución general de la ecuación diferencial lineal homogénea
\begin{equation}
y^{\prime\prime}-4y=0
\label{Ecu2}
\end{equation}
Sustituyendo, $y=k_{0}e^{st}$ en (\ref{Ecu2}), obtenemos la ecuación característica
\begin{equation}
s^{2}-4=0
\end{equation}
de lo cual, se obtienen las raíces $s_{1}=2,\,s_{2}=-2$. Entonces, sabemos que la solución 
general de (\ref{Ecu2}) es de la forma
\begin{equation}
y_{h}(t)=c_{1}e^{2t}+c_{2}e^{-2t}
\label{Ecu3}
\end{equation}
Como sabemos que para cualesquiera dos constantes reales $c_{1}$ y $c_{2}$ (Ecu3) es solución 
de la ecuación diferencial lineal homogénea (\ref{Ecu2}), esto es,
\begin{equation}
y_{h}^{\prime\prime}-4y_{h}=0
\label{Ecu4}
\end{equation}
la solución del problema de valor inicial (\ref{Ecu0}), debe ser de la forma
\begin{equation}
y(t)=y_{h}(t)+y_{p}(t)
\label{Ecu5}
\end{equation}
Sustituyendo (\ref{Ecu5}) en (\ref{Ecu0}) se tiene que
\begin{equation}
y_{h}^{\prime\prime}-4y_{h}+y_{p}^{\prime\prime}-4y_{p}=12t
\label{Ecu6}
\end{equation}
Usando el hecho indicado por la ecuación (\ref{Ecu4}), de (\ref{Ecu6}) encontramos que 
\begin{equation}
y_{p}^{\prime\prime}-4y_{p}=12t
\label{Ecu7}
\end{equation}
Una forma sistemática de encontrar $y_{p}(t)$, tendrá que esperar hasta que revisemos 
el tema de sistemas de ecuaciones y el método de variación de parámetros. Por ahora, sólo 
haremos una deducción medio ingeniosa, medio ``mágica'' de la solución para (\ref{Ecu7}). 
Primero notamos que si $y_{p}$ es una constante que multiplica a $t$, la segunda derivada de 
$y_{p}$ es igual a cero. De tal manera que si $y_{p}=c_{3}t$, (donde $c_{3}$ es una constante), $y_{p}^{\prime\prime}=0$; y se 
puede despejar $y_{p}$ de (\ref{Ecu7}):
\begin{eqnarray*}
y_{p}^{\prime\prime}-4y_{p}&=&12t\\
0-4y_{p}(t)&=&12t\\
-4y_{p}(t)&=&12t\\
y_{p}(t)&=&-3t
\label{Ecu8}
\end{eqnarray*}
De lo cual concluimos que
\begin{equation}
y_{p}(t)=-3t
\label{Ecu9}
\end{equation}
Sustituyendo (\ref{Ecu3}) y (\ref{Ecu9}) en (\ref{Ecu5}) tenemos
\begin{equation}
y(t)=c_{1}e^{2t}+c_{2}e^{-2t}-3t
\label{Ecu10}
\end{equation}
Finalmente para encontrar las constantes $c_{1}$ y $c_{2}$ se utiliza el siguiente sistema de 
ecuaciones
\begin{eqnarray}
\left.c_{1}e^{2t}+c_{2}e^{-2t}-3t\right|_{t=0}&=&y(0)\label{Ecu11}\\
\left.2c_{1}e^{2t}-2c_{2}e^{-2t}-3\right|_{t=0}&=&y^{\prime}(0)\label{Ecu12}
\end{eqnarray}
Con los valores iniciales $y(0)=4$, $y^{\prime}(0)=1$, el sistema (\ref{Ecu11})-(\ref{Ecu12}) se 
convierte en
\begin{eqnarray}
c_{1}+c_{2}&=&4\label{Ecu13}\\
2c_{1}-2c_{2}-3&=&1\label{Ecu14}
\end{eqnarray}
es decir,
\begin{eqnarray}
c_{1}+c_{2}&=&4\label{Ecu13}\\
2c_{1}-2c_{2}&=&4\label{Ecu14}
\end{eqnarray}
Resolviendo este sistema de ecuaciones encontramos que $c_{1}=3$ y $c_{2}=1$. Con lo que concluimos 
que la solución al problema de valor inicial que estamos resolviendo es
\begin{equation}
y(t)=3e^{2t}+e^{-2t}-3t
\end{equation}
Dicho de forma más elegante, la solución al problema de valor inicial: encontrar $y(t)$ tal que
\begin{equation}
y^{\prime\prime}-4y=12t,\hspace{2cm}y(0)=4,\,\,y^{\prime}(0)=1.
\label{Ecu15}
\end{equation}
es 
\begin{equation}
y(t)=3e^{2t}+e^{-2t}-3t
\label{Ecu16}
\end{equation}


\ \\
REF. Dennis G. Zill, Michael Cullen, Differential Equations with Boundary Value Problems
\end{document}

