\documentclass[12pt]{article}
\usepackage[total={18cm,22cm},top=1cm, left=2cm]{geometry}
\usepackage[spanish]{babel}
\usepackage[utf8]{inputenc}
\def\TresOp#1#2#3{
\begin{itemize}
\item[A)] #1
\item[B)] #2
\item[C)] #3
\end{itemize}
}
\def\LaFormulaFXYA{
$\int P(x,y)dx+\int\left[Q(x,y)-\frac{\partial}{\partial y}\int P(x,y)dx\right]dy$
}
\def\LaFormulaFXYB{
$\int P(x,y)dy+\int\left[Q(x,y)+\frac{\partial}{\partial y}\int P(x,y)dx\right]dy$
}
\def\LaFormulaFXYC{
$\int P(x,y)dy+\int\left[Q(x,y)-\frac{\partial}{\partial y}\int P(x,y)dx\right]dx$
}

\begin{document}
\begin{enumerate}
% 1
\item Son ejemplos de ecuaciones diferenciales para los que se conocen métodos de solución
\TresOp{Ecuaciones diferenciales funcionales.}
{Ecuaciones estocásticas y ecuaciones separables.}
{Ecuaciones de variables separables y ecuaciones homogéneas.}
{\tiny R=C}
% 2
\item Si una ecuación diferencial se puede escribir en la forma $f_{1}(x)dx+f_{2}(y)dy=0$ se dice que 
es una ecuación de tipo
\TresOp{Variables separables}{Homogénea}{Estocástica}
{\tiny R=A}
% 3
\item La ecuación diferencial $\frac{dy}{dx}+e^{x}y=e^{x}y^{2}$ se puede reescribir como
\TresOp{$\frac{dy}{dx}+e^{x}(y+y^{2})=0$}
{$\frac{dy}{dx}+e^{x}(y-y^{2})=0$}
{$\frac{dy}{dx}+e^{x}(y^{2}-y)=0$}
{\tiny R=B}
% 4
\item ?`A qué es igual la integral $\int e^{x}dx$?
\TresOp{$e^{x}$}
{$2e^{x}$}
{$e^{2x}$}
{\tiny R=A}
% 5
\item ?`A qué es igual la integral $\int\frac{dy}{y-y^{2}}$?
\TresOp{$\ln\frac{y}{1-y}$}
{$\ln\left|\frac{1}{y(1-y)}\right|$}
{$\ln\left|\frac{y}{1-y}\right|$}
{\tiny R=C}
\newpage
% 6
\item ?`Cuál es la solución de la ecuación $\frac{dy}{y-y^{2}}+e^{x}dx=0$?
\TresOp{$\frac{y^{2}}{2}-\frac{y^{3}}{3}+e^{x}=C$}
{$\ln\left|\frac{y}{1-y}\right|+e^{x}=C$}
{$\ln\left|\frac{y}{1-y}\right|+e^{2x}=C$}
{\tiny R=B}
% 7
\item Si una ecuación diferencial se puede reescribir en la forma 
$\frac{dy}{dx}=F\left(\frac{y}{x}\right)$, se dice que es una ecuación
\TresOp{diferencial de variables separables}
{diferencial homogénea}
{diferencial exacta}
{\tiny R=B}
% 8
\item La ecuación diferencial $(x^{2}+y^{2})dx+2xydy$ se puede reescribir como
\TresOp{$\frac{dy}{dx}=-\frac{1+(y/x) ^{2}}{2(y/x)}$}
{$\frac{dy}{dx}=\frac{1+(y/x) ^{2}}{2(y/x)}$}
{$\frac{dy}{dx}=-\frac{(y/x) ^{2}}{2(y/x)+1}$}
{\tiny R=A}
% 9
\item ?`A qué es igual la integral $\int\frac{dx}{x}$?
\TresOp{$\ln\left(\frac{1}{x}\right)$}
{$\ln x^{2}$}
{$\ln\left|x\right|$}
{\tiny R=C}
% 10
\item ?`A qué es igual la integral $\int\frac{2vdv}{1+3v ^{2}}$?
\TresOp{$\ln\left|1+3v^{2}\right|$}
{$3\ln\left|1+3v^{2}\right|$}
{$\frac{1}{3}\ln\left|1+3v^{2}\right|$}
{\tiny R=C}
\newpage
%11
\item ?`Cuál es la solución de la ecuación $(x^{2}+y^{2})dx+2xydy$?
\TresOp{$x(x^{2}+3y^{2})=C$}
{$x^{2}(x+3y^{2})=C$}
{$x(2x+3y^{2})=C$}
{\tiny R=A}
%12
\item Una ecuación de la forma $P(x,y)dx+Q(x,y)dy=0$ para la cual se cumple que 
$\frac{\partial P}{\partial y}=\frac{\partial Q}{\partial x}$ es
\TresOp{ecuación diferencial homogénea}
{ecuación diferencial exacta}
{ecuación diferencial de variables separables}
{\tiny R=B}
%13
\item Si $\frac{\partial P}{\partial y}=\frac{\partial Q}{\partial x}$ una fórmula para 
la solución de la ecuación diferencial $P(x,y)dx+Q(x,y)dy=0$ es
\TresOp{$F(x,y)=C$, donde $F(x,y)=$\LaFormulaFXYA}
{$F(x,y)=C$, donde $F(x,y)=$\LaFormulaFXYB}
{$F(x,y)=C$, donde $F(x,y)=$\LaFormulaFXYC}
{\tiny R=A}
%14
\item La solución de la ecuación diferencial $(2xy+1)dx+(x^{2}+4y)dy=0$ es
\TresOp{$xy^{2}+x^{2}+2y=C$}
{$x^{2}y+x^{2}+2y=C$}
{$x^{2}y+x+2y^{2}=C$}
{\tiny R=C}
%15
\item Si consideramos que $\frac{\partial F}{\partial x}=2xy+1$, y lo integramos con respecto 
a $x$ (tratando a $y$ como constante) el resultado es
\TresOp{$x^{2}y+x+c_{1}(y)$}
{$x^{2}y+x-1$}
{$xy^{2}+x+c_{1}(y)$}
{\tiny R=A}
\newpage
%16
\item Si consideramos que $\frac{\partial F}{\partial y}=x^{2}+4y$, y lo integramos con respecto 
a $y$ (tratando a $x$ como constante) el resultado es
\TresOp{$x^{2}y+2y+c_{2}(x)$}
{$x^{2}y+2y^{2}+c_{2}(x)$}
{$x^{2}y+2y^{2}+1$}
{\tiny R=B}
%17
\item La ecuación diferencial $x^{2}\frac{dy}{dx}-y^{2}=x^{2}y\frac{dy}{dx}$ se puede reescribir en 
la forma
\TresOp{$x^{2}(1-y)\frac{dy}{dx}-y^{2}=0$}
{$x^{2}y\frac{dy}{dx}-y^{2}=0$}
{$x^{2}(1-y)\frac{dy}{dx}+y^{2}=0$}
{\tiny R=A}
%18
\item La solución de la ecuación $(\frac{1}{y^{2}}-\frac{1}{y})dy-\frac{dx}{x^{2}}=0$
\TresOp{$\frac{1}{y}+\ln\left|y\right|-\frac{1}{x}=C$}
{$-\frac{1}{y}-\ln\left|y\right|+\frac{1}{x}=C$}
{$\ln\left|\frac{1}{y}\right|-\ln\left|y\right|+\frac{1}{x}=C$}
{\tiny R=B}
%19
\item Para resolver la ecuación diferencial homogénea $\frac{dy}{dx}=\frac{xy-y^{2}}{x^{2}}$ 
podemos usar la función
\TresOp{$F(v)=v^{2}+v$}
{$F(v)=v-v^{2}$}
{$F(v)=\frac{1}{v-v^{2}}$}
{\tiny R=B}
%20
\item La solución de la ecuación $\frac{dx}{x}+\frac{dv}{v^{2}}=0$ es
\TresOp{$\ln\left|x\right|-\frac{1}{v^{2}}=C$}
{$\ln\left|x\right|+\frac{1}{v}=C$}
{$\ln\left|x\right|-\frac{1}{v}=C$}
{\tiny R=C}
\newpage
%21
\item La solución de la ecuación diferencial homogénea $\frac{dy}{dx}=\frac{xy-y^{2}}{x^{2}}$ es
\TresOp{$\ln\left|x\right|-\frac{x}{y}=C$}
{$\ln\left|x\right|+\frac{x}{y}=C$}
{$e^{x}-\frac{y}{x}$}
{\tiny R=A}
%22
\item Si consideramos que $\frac{\partial F}{\partial x}=e^{x}+1$, y lo integramos con respecto 
a $x$ (tratando a $y$ como constante) el resultado es
\TresOp{$e^{x}+x+c_{1}(y)$}
{$e^{x}+1+c_{1}(y)$}
{$\ln\left|x\right|+c_{1}(y)$}
{\tiny R=A}
%23
\item Si consideramos que $\frac{\partial F}{\partial y}=1$, y lo integramos con respecto 
a $y$ (tratando a $x$ como constante) el resultado es
\TresOp{$y+c_{2}(y)$}
{$y+c_{2}(x)$}
{$xy+c_{2}(x)$}
{\tiny R=B}
%24
\item La solución de la ecuación diferencial exacta $(e^{x}+1)dx+dy=0$ es
\TresOp{$\ln\left|x\right|+x+y=C$}
{$e^{x}+x+y=C$}
{$e^{x}+x+\frac{1}{y}=C$} 
{\tiny R=B}
%25
\item La ecuación $(y\cos xy+2x)dx+(x\cos xy)dy=0$ es
\TresOp{una ecuación diferencial de variables separables.}
{una ecuación diferencial exacta.}
{una ecuación diferencial homogénea.}
{\tiny R=B}
\newpage
%26
\item Si consideramos que  $\frac{\partial F}{\partial x}=y\cos xy+2x$, y lo integramos con respecto 
a $x$ (tratando a $y$ como constante) el resultado es 
\TresOp{$y\cos xy+2x$}
{$\sin xy+x^{2}+c_{1}(y)$}
{$\sin xy+x^{2}+c_{1}(y)$}
{\tiny R=C}
%27
\item Si consideramos que  $\frac{\partial F}{\partial y}=x\cos xy$, y lo integramos con respecto 
a $y$ (tratando a $x$ como constante) el resultado es
\TresOp{$\sin xy+c_{2}(y)$}
{$\sin xy+c_{2}(x)$}
{$x\cos xy$}
{\tiny R=B}
%28
\item ?`Cuál es la solución de la ecuación diferencial $(y\cos xy+2x)dx+(x\cos xy)dy=0$?
\TresOp{$\sin xy+x^{2}=C$}
{$\cos xy+2x$}
{$\cos xy$}
{\tiny R=A}
%29
\item La ecuación $xdy-ydx=0$ puede ser convertida a exacta utilizando el factor
\TresOp{$\frac{1}{x^{2}}$}
{$x$}
{$\frac{1}{x}$}
{\tiny R=A}
%30
\item Para una ecuación de la forma $M(t,y)+N(t,y)\frac{dy}{dt}=0$, si la función 
$\frac{\frac{\partial M}{\partial y}-\frac{\partial N}{\partial t}}{N}$ depende únicamente de $t$, 
el factor integrante se calcula con la fórmula
\TresOp{$e^{\int\frac{\left[\frac{\partial M}{\partial y}-\frac{\partial N}{\partial t}\right]dt}{N}}$}
{$\frac{\frac{\partial M}{\partial y}-\frac{\partial N}{\partial t}}{N}$}
{$\frac{d}{dt}\left(\frac{\frac{\partial M}{\partial y}-\frac{\partial N}{\partial t}}{N}\right)$}
{\tiny R=A}
\newpage
%31
\item ?`Cuál es el factor integrante para ecuación $y+\frac{dy}{dt}=0$?
\TresOp{$\frac{t^{2}}{2}$}
{$e^{t^{2}}$}
{$e^{t}$}
{\tiny R=C}
%32
\item ?`Cuál es la solución general de la ecuación $y+\frac{dy}{dt}=0$?
\TresOp{$y=C\ln\left|t\right|$}
{$y=Ce^{-t}$}
{$y=Ce^{t}$}
{\tiny R=B}
%33
\item Si $R$ y $L$ son dos constantes positivas, ?`Cuál es la solución general de la ecuación 
diferencial $\frac{di}{dt}+\frac{R}{L}i(t)=0$?
\TresOp{$i(t)=I_{0}(e^{-Rt/L}+\ln\left|Rt/L\right|)$ donde $I_{0}$ es una constante.}
{$i(t)=I_{0}\ln\left|Rt/L\right|$ donde $I_{0}$ es una constante.}
{$i(t)=I_{0}e^{-Rt/L}$ donde $I_{0}$ es una constante.}
{\tiny R=C}
%34
\item Si $R$ y $C$ son dos constantes positivas, considere la ecuación diferencial 
$\frac{dv}{dt}+\frac{v}{RC}=0$, si substituimos $v=k_{0}e^{st}$ (donde $k_{0}$ es una constante) 
en la ecuación diferencial obtenemos
\TresOp{$(s^{2}+\frac{1}{RC})k_{0}e^{st}=0$}
{$(s+\frac{1}{RC})k_{0}e^{st}=0$}
{Niguno de los anteriores.}
{\tiny R=B}
%35
\item La ecuación $a_{n}(t)\frac{d^{n}y}{dt^{n}}+a_{n-1}(t)\frac{d^{n-1}y}{dt^{n-1}}+\cdots
a_{1}(t)\frac{dy}{dt}+a_{0}(t)y(t)=g(t)$ cuando $g(t)=0$ es
\TresOp{una ecuación diferencial lineal homogénea.}
{una ecuación diferencial lineal no homogénea.}
{una ecuación diferencial no lineal homogénea.}
{\tiny R=A}
\newpage
%36
\item Si $R$, $L$ y $C$ son constantes positivas, al sustituir $v(t)=K_{0}e^{st}$ (donde $K_{0}$ y 
$s$ son constantes) en la ecuación $LC\frac{d^{2}v}{dt^{2}}+RC\frac{dv}{dt}+v(t)=0$ se obtiene
\TresOp{$(LCs^{2}+RCs+1)K_{0}e^{st}=0$}
{$\frac{\partial M}{\partial y}=\frac{\partial N}{\partial x}$}
{$(LCs^{2}+RCs+1)K_{0}e^{-st}=0$}
{\tiny R=A}
%37
\item Si $a$, $b$ y $c$ son constantes, para la ecuación  $a\frac{d^{2}y}{dt^{2}}+b\frac{dy}{dt}+cy=0$ 
la ecuación $as^{2}+bs+c=0$ recibe el nombre de 
\TresOp{ecuación no lineal homogénea.}
{ecuación lineal.}
{ecuación característica.}
{\tiny R=C}
%38
\item Para la ecuación diferencial $LC\frac{d^{2}v}{dt^{2}}+RC\frac{dv}{dt}+v(t)=0$, ?`Cuál es la 
forma de la solución cuando las raíces de la ecuación $LCs^{2}+RCs+1=0$, $s_{1}$ y $s_{2}$ son 
reales y diferentes?
\TresOp{$V_{1}e^{s_{1}t}+V_{2}te^{s_{2}t}$, donde $V_{1}$ y $V_{2}$ son constantes.}
{$V_{1}e^{s_{1}t}+V_{2}e^{s_{2}t}$, donde $V_{1}$ y $V_{2}$ son constantes.}
{$V_{1}\ln(s_{1}t)+V_{2}\ln(s_{2}t)$, donde $V_{1}$ y $V_{2}$ son constantes.} 
{\tiny R=B}
%39
\item Para la ecuación diferencial $LC\frac{d^{2}v}{dt^{2}}+RC\frac{dv}{dt}+v(t)=0$, ?`Cuál es la 
forma de la solución cuando las raíces de la ecuación $LCs^{2}+RCs+1=0$, $s_{1}$ y $s_{2}$ son 
reales e iguales ($s_{1}=s_{2}=s$)?
\TresOp{$V_{1}e^{st}+V_{2}te^{st}$, donde $V_{1}$ y $V_{2}$ son constantes.}
{$V_{1}e^{st}+V_{2}e^{st}$, donde $V_{1}$ y $V_{2}$ son constantes.}
{$V_{1}\ln(st)+V_{2}\ln(st)$, donde $V_{1}$ y $V_{2}$ son constantes.}
{\tiny R=A}
%40
\item Para la ecuación diferencial $LC\frac{d^{2}v}{dt^{2}}+RC\frac{dv}{dt}+v(t)=0$, ?`Cuál es la 
forma de la solución cuando las raíces de la ecuación $LCs^{2}+RCs+1=0$, $s_{1}$ y $s_{2}$ son 
complejas conjugadas ($s_{1}=\alpha+j\beta,\,\,s_{2}=\alpha-j\beta$ donde $\alpha>0$ y $\beta>0$)?
\TresOp{$V_{1}e^{\alpha t}+V_{2}te^{\beta t}$, donde $V_{1}$ y $V_{2}$ son constantes.}
{$V_{1}e^{\alpha t}+V_{2}e^{\beta t}$, donde $V_{1}$ y $V_{2}$ son constantes.}
{$e^{\alpha t}(V_{1}\cos\beta t+V_{2}\sin\beta t)$, donde $V_{1}$ y $V_{2}$ son constantes.}
{\tiny R=C}
\newpage
%41
\item Para el problema de valor inicial $0.00001\frac{d^{2}y}{dt^{2}}+0.01\frac{dy}{dt}+y(t)=0$ 
sujeto a $y(0)=5$ y $y^{\prime}(0)=1000$. ?`Cuáles son las raíces de $0.00001s^{2}+0.01s+1=0$?
\TresOp{$-50,\,\,-100$}
{$-112.70,\,\,-887.29$}
{$-112.70,\,\,-100.25$}
{\tiny R=B}
%42
\item Para el problema de valor inicial $0.00001\frac{d^{2}y}{dt^{2}}+0.01\frac{dy}{dt}+y(t)=0$ 
sujeto a $y(0)=5$ y $y^{\prime}(0)=1000$. ?`Cuáles son los valores de las constantes 
$V_{1}$ y $V_{2}$ de la solución al problema $v(t)=V_{1}e^{s_{1}t}+V_{2}e^{s_{2}t}$?
\TresOp{$10.3407,\,\,-5.3407$}
{$7.0184,\,\,-2.0184$}
{$15.6304,\,\,-10.6304$}
{\tiny R=B}
%43
\item Para el problema de valor inicial $0.000025\frac{d^{2}y}{dt^{2}}+0.01\frac{dy}{dt}+y(t)=0$ 
sujeto a $y(0)=5$ y $y^{\prime}(0)=1000$. ?`Cuáles son las raíces de $0.000025s^{2}+0.01s+1=0$?
\TresOp{$-50+j100,\,\,-50-j100$}
{$-112.70,\,\,-887.29$}
{$-200.00,\,\,-200.00$}
{\tiny R=C}
%44
\item Para el problema de valor inicial $0.000025\frac{d^{2}y}{dt^{2}}+0.01\frac{dy}{dt}+y(t)=0$ 
sujeto a $y(0)=5$ y $y^{\prime}(0)=1000$. ?`Cuáles son los valores de las constantes 
$V_{1}$ y $V_{2}$ de la solución al problema $v(t)=V_{1}e^{s_{1}t}+V_{2}te^{s_{2}t}$?
\TresOp{$5.0,\,\,100.0$}
{$5.0,\,\,300.0$}
{$5.0,\,\,2000.0$}
{\tiny R=C}
%45
\item Para el problema de valor inicial $0.0001\frac{d^{2}y}{dt^{2}}+0.01\frac{dy}{dt}+y(t)=0$ 
sujeto a $y(0)=5$ y $y^{\prime}(0)=1000$. ?`Cuáles son las raíces de $0.0001s^{2}+0.01s+1=0$?
\TresOp{$-50+j86.6025,\,\,-50-j86.6025$}
{$-112.70,\,\,-887.29$}
{$-200.00,\,\,-200.00$}
{\tiny R=A}
\newpage
%46
\item Para el problema de valor inicial $0.000025\frac{d^{2}y}{dt^{2}}+0.01\frac{dy}{dt}+y(t)=0$ 
sujeto a $y(0)=5$ y $y^{\prime}(0)=1000$. ?`Cuáles son los valores de las constantes 
$V_{1}$ y $V_{2}$ de la solución al problema $v(t)=e^{\alpha t}(V_{1}\cos\beta t+V_{2}\sin\beta t)$?
\TresOp{$5.0,\,\,14.4337$}
{$5.0,\,\,30.4337$}
{$5.0,\,\,2000.0$}
{\tiny R=A}
%47
\item Sobre el intervalo $(0,\infty)$, el conjunto de funciones 
$\left\{f_{1},\,f_{2}\right\}=\left\{2t,\,\left|t\right|\right\}$ es
\TresOp{linealmente dependientes.}
{linealmente independientes.}
{ninguno de los anteriores.}
{\tiny R=A}
%48
\item ?`Cuál de las siguientes funciones es una solución de la ecuación $y^{\prime\prime}-y=0$?
\TresOp{$\cos t$}
{$e^{t}$}
{$e^{t}\cos t$}
{\tiny R=B}
%49
\item Usando el método de reducción de orden, partiendo de la solución $y_{1}=e^{t}$ de 
$y^{\prime\prime}-y=0$, se propone encontrar $y=ue^{t}, $ ?`Cuál de las siguientes ecuaciones es la que se obtiene para encontrar la función $w$? (donde $w=u^{\prime}$) 
\TresOp{$w^{\prime}+2w=0$}
{$\frac{d}{dt}(e^{t}w)=0$}
{$w^{\prime}-2w=0$}
{\tiny R=A}
%50
\item ?`Cuál es el Wronskiano del conjunto de funciones 
$\left\{y_{1},y_{2}\right\}=\left\{t^{2},t^{2}\ln(t)\right\}$ sobre el intervalo $(0,\infty)$?
\TresOp{$t^{3}$}
{$1$}
{$t\ln(t)$}
{\tiny R=A}
\end{enumerate}
\end{document}