\documentclass[12pt]{article}
\title{Crecimiento Continuo}
\author{Ing. Lamberto Maza Casas}
%\date{\today}
\date{Marzo, 2020}
\begin{document}
\maketitle
En este documento se presentan algunos comentarios sobre una ecuaci\'{o}n diferencial  
para modelar el crecimiento continuo de una cantidad.
\section{Crecimiento Continuo}
Decimos que una cantidad $C$ tiene {\bf crecimiento continuo} a una tasa dada en porcentaje 
$p>0$ si existe un n\'{u}mero f\/ijo $T>0$ tal que para cualesquiera n\'{u}meros reales 
$t\in(-\infty,+\infty)$ y $|h|\in(0,T]$ se cumple que
\begin{equation}
c(Tt+h)=c(Tt)+\frac{h}{T}rc(Tt),\mbox{\ donde\ }r=\frac{p}{100}
\end{equation}
Tambi\'{e}n decimos que la cantidad $c$ crece $p$ ``por ciento'' cada $T$ unidades de tiempo; es decir, 
$c$ crece con un factor de proporci\'{o}n $r=p/100$ cada $T$ unidades de tiempo.
\section{Obtenci\'{o}n de la ecuaci\'{o}n diferencial de cre\-ci\-mien\-to continuo}
Supongamos que la cantidad $y$ tiene crecimiento continuo a una tasa $p$. Entonces, sabemos que 
existe $T>0$ para el cual cuando $t$ es un n\'{u}mero real y $|h|\in(0,T]$, entonces
\begin{equation}
y(Tt+h)=y(Tt)+\frac{h}{T}ry(Tt),\mbox{\ donde\ }r=\frac{p}{100}
\end{equation}
Lo cual signif\/ica que
\begin{equation}
\frac{y(Tt+h)-y(Tt)}{h}=\frac{r}{T}y(Tt)
\end{equation}
Si hacemos el cambio de variable $x=Tt$, obtenemos
\begin{equation}
\frac{y(x+h)-y(x)}{h}=\frac{r}{T}y(x)
\end{equation}
Si ahora tomamos el l\'{i}mite cuando $h\rightarrow 0$, obtenemos
\begin{equation}
\lim_{h\rightarrow 0}\frac{y(x+h)-y(x)}{h}=\lim_{h\rightarrow 0}\frac{r}{T}y(x)=\frac{r}{T}y(x)
\end{equation}
El lado derecho es la forma en que se calcula la derivada de $y$ con respecto a $x$. Por lo que hemos 
llegado a la ecuaci\'{o}n diferencial
\begin{equation}
\frac{dy}{dx}=\frac{r}{T}y(x)
\end{equation}
Agregando la condici\'{o}n inicial $y(x_{0})=y_{0}$ (donde $x_{0}=Tt_{0}$) obtenemos un problema de valor inicial (o problema 
de Cauchy):
\begin{equation}
\frac{dy}{dx}=\frac{r}{T}y(x),\mbox{\ donde\ }y(x_{0})=y_{0}
\end{equation}
La soluci\'{o}n general de este problema de valor inicial est\'{a} dada por
\begin{equation}
y(x)=y(x_{0})e^{\frac{r}{T}(x-x_{0})}
\end{equation}
Susttituyendo nuevamente el cambio de variable $x=Tt$ se llega a
\begin{equation}
y(Tt)=y(Tt_{0})e^{r(t-t_{0})}
\end{equation}
donde $y(Tt_{0})$ es cualquier valor real constante. Si def\/inimos la funci\'{o}n del tiempo $z(t)$ 
como
\begin{equation}
z(t)=y(Tt)=y(Tt_{0})e^{r(t-t_{0})},\mbox{\ n\'{o}tese que }z(t_{0})=y(Tt_{0})
\end{equation}
obtenemos la funci\'{o}n que describe el crecimiento continuo de una cantidad cuando esta crece a 
una tasa fija $p$ (en pocentaje) cada $T$ unidades de tiempo. ?`En cu\'{a}nto tiempo se duplica una 
cantidad que crece a una tasa fija $p$ (en porcentaje) cada $T$ unidades de tiempo? En otras palabras, 
?`cu\'{a}l es el valor de $t-t_{0}$ para el cual $z(t)=2z(t_{0})$?
Para responder a la pregunta escribimos
\begin{eqnarray}
z(t)&=&2z(t_{0})\\
z(t_{0})e^{r(t-t_{0})}&=&2z(t_{0})\\
e^{r(t-t_{0})}&=&2\\
r(t-t_{0})&=&ln(2)\\
t-t_{0}&=&\frac{ln(2)}{r}\label{EcQuince}
\end{eqnarray}
Si se expresa el factor de proporcionalidad en funci\'{o}n de la tasa de crecimiento $p$ en porcentaje 
obtenemos $r=p/100$, sustituyendo esto en la ecuaci\'{o}n (\ref{EcQuince}) llegamos a
\begin{equation}
t-t_{0}=\frac{100ln(2)}{p}=\frac{100(0.6931)}{p}\approx\frac{70}{p}
\end{equation}
\end{document}
