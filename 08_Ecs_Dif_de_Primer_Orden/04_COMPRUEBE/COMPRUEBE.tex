\documentclass[12pt]{article}
\usepackage{tikz}
\usepackage{pgf-umlcd}
\begin{document}
 \section{COMPRUEBE}
 Considerando $P(x,y)dx+Q(x,y)dy=0$ como
 \begin{equation}
 (2xy+1)dx+(x^{2}+4y)dy=0.
\end{equation}
Dado que
\begin{equation}
 \frac{\partial P}{\partial y}=
\frac{\partial Q}{\partial x}=2x,
\end{equation}
 COMPRUEBE que la sustituci\'{o}n de las expresiones $P(x,y)=2xy+1$ y $Q(x,y)=x^{2}+4y$ en la f\'{o}rmula
 \begin{equation}
 F(x,y)=\int P(x,y)dx+
\int\left[Q(x,y)-
\frac{\partial}{\partial y}\int P(x,y)dx\right]dy.
\label{Ecu77-7}
\end{equation}
 da como resultado
\begin{equation}
 F(x,y)=x^{2}y+x+2y^{2}.
\end{equation}
Comprobaci\'{o}n
\begin{eqnarray}
 \int P(x,y)dx&=&\int (2xy+1)dx\\
 &=&2\int xydx+\int dx\\
 &=&2(\frac{x^{2}}{2}y)+x\\
 &=&x^{2}y+x
 \label{EcuN}
\end{eqnarray}
Usando (\ref{EcuN})
\begin{eqnarray}
 \frac{\partial}{\partial y}\int P(x,y)dx&=&\frac{\partial}{\partial y}\left[x^{2}y+x\right]\\
 &=&x^{2}
\end{eqnarray}
Ahora calculamos la integral
\begin{eqnarray}
\int\left[Q(x,y)-
\frac{\partial}{\partial y}\int P(x,y)dx\right]dy&=&\int\left[(x^{2}+4y)-(x^{2})\right]dy\\
&=&\int4ydy\\
&=&2y^{2}
\label{EcuN+3}
\end{eqnarray}
Por lo tanto, sustituyendo (\ref{EcuN}) y (\ref{EcuN+3}) en (\ref{Ecu77-7})
\begin{equation}
 F(x,y)=x^{2}y+x+2y^{2}
\end{equation}
que es lo que se quer\'{i}a comprobar.
\end{document}
